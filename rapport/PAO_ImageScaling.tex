\documentclass[12pt, a4paper]{article}
\usepackage[utf8]{inputenc}
%\usepackage[french]{babel}
\usepackage{amsfonts}
\usepackage{graphicx}
\usepackage{fancyhdr}
\usepackage{url}
\usepackage[usenames,dvipsnames]{color}

\usepackage[top=3.3cm, bottom=3.3cm, left=2cm, right=2cm]{geometry}

\renewcommand{\baselinestretch}{1.2}

\pagestyle{fancy}
\usepackage{lastpage}
\renewcommand\headrulewidth{1pt}
\fancyhead[L]{Image UpScaling}
\fancyhead[R]{INSA de Rouen}
\renewcommand\footrulewidth{1pt}
\fancyfoot[C]{\textbf{Page \thepage/\pageref{LastPage}}}
\fancyfoot[R]{\includegraphics[width=0.10\textwidth]{Images/Logo_INSA.png}}

\title{Rapport de projet}
\author{Ophélie Guenoux \\Olivier Petit}
\date{Janvier 2016}

\begin{document}

\makeatletter
\begin{titlepage}
  \begin{center}
      \includegraphics[width=0.20\textwidth]{Images/Logo_INSA.png}
      \hfill
      \includegraphics[width=0.25\textwidth]{Images/logoasi.png}\\
    \vspace{1cm}
		\Huge \underline{\@title} 
			\\ \textsc{Image UpScaling}
			\\ \Large Projet d'Approfondissement et d'Ouverture
			\vspace{1cm}
			\begin{figure}[h!]
				\centering
				%\includegraphics[width=0.7\textwidth]{Images/radcul.jpg}
			\end{figure}
	\vspace{1cm}
	\end{center}
	\raggedright
	\large Ophélie Guenoux \hfill Encadrant : M.Chatelain \& M.Herault
	\\Olivier Petit \hfill INSA Rouen ASI4
	
\end{titlepage}

\newpage
\tableofcontents
\newpage
\section*{Introduction}

\section{État de l'art}
\section{Choix de la librairie}
Pour effectuer de l'agrandissement d'image, nous souhaitons utilisez des réseaux de neurones. De ce fait, nousavons commencé à nous renseigner sur les différentes librairies. 

 Nous THeoano librairie machine learning reconnu pour sa stabilité
\subsection{Utilisation de \emph{YAML} avec Python}

\section{Les différentes étapes du projet}
	\subsection{Choix et construction du Dataset}

	\subsection{Utilisation de Patchs}
	% mettre des images en entier sur le réseau -> beaucoup trop lent. Du coup réduction de la taille du réseau en ne passant que les patchs
	% deux nouveaux algo pour découper l'image en patch et pour la reformer
	\subsection{Structure du réseau}
	% nb couche
	\subsubsection{Changement des fonctions d'activation}
	% erreur de domaine de nos données, SIgmoid prend entre -inf et +inf pour ressortir en 0 et 1 ; 
	% ajouter un shéma pour expliquer les différentes couches 
\section{Problèmes rencontrés}
	\subsection{Le problème du pas}
	% Avant nous baissions tout simplement le learning rate ds une certaines fourchette d'epochs
	% tente d'utiliser un algo plus "intelligent" qui regarde 1 channel +val de sortie et par rapport à cette valeur fait augmenter ou dim le LR
	\subsection{Upscalling-RGB}
	% tentative de faire un réseau de neurones capable de faire de l'upscalling avec des images en couleur. 
	% -> problème notre image de sortie est en niveau de bleu principalement

\section*{Conclusion}
\end{document}
